%%%%%%%%%%%%%%%%%%%%%%%%%%%%%%%%%%%%%%%%%
% Developer CV
% LaTeX Template
% Version 1.0 (28/1/19)
%
% This template originates from:
% http://www.LaTeXTemplates.com
%
% Authors:
% Jan Vorisek (jan@vorisek.me)
% Based on a template by Jan Küster (info@jankuester.com)
% Modified for LaTeX Templates by Vel (vel@LaTeXTemplates.com)
%
% License:
% The MIT License (see included LICENSE file)
%
%%%%%%%%%%%%%%%%%%%%%%%%%%%%%%%%%%%%%%%%%

%----------------------------------------------------------------------------------------
%   PACKAGES AND OTHER DOCUMENT CONFIGURATIONS
%----------------------------------------------------------------------------------------

\documentclass[9pt]{developercv} % Default font size, values from 8-12pt are recommended

\begin{document}
\begin{minipage}[t]{1\textwidth} % 45% of the page width for name
    \vspace{-\baselineskip} % Required for vertically aligning minipages
    
    % If your name is very short, use just one of the lines below
    % If your name is very long, reduce the font size or make the minipage wider and reduce the others proportionately
    \colorbox{black}{{\HUGE\textcolor{white}{\textbf{\MakeUppercase{Kiron Roy}}}}} \hspace{.3 cm} \large\texttt{732-241-1722\slashsep\href{mailto:kroy.xc@gmail.com}{kroy.xc@gmail.com}\slashsep\href{https://github.com/kroy}{github.com/kroy}}
    
    \vspace{6pt}
    
    %{\huge Web App Architect} % Career or current job title
\end{minipage}

\vspace{0.5cm}

\parbox{1\textwidth}{To my (hopefully) future coworkers,}\\
\vspace{0.5cm}\\
\parbox{1\textwidth}{\hspace{0.75cm}In the early 2010’s, when I first entered the tech industry fresh out of college, it was easy to believe the “have your cake and eat it” pitch from most tech companies. With technology, we could make the world better, and make a ton of money while doing it. Tech companies at the time operated as kinds of wonder machines where capital went in and [social|shareholder|employee] value spewed out the other end. Tech bosses were masters at trading in the easy politics of hope and change, at once reassuring the public that they weren’t like the evil banks who’d torpedoed the economy, nor were they like those Occupy Wallstreeters who wanted to… occupy stuff or whatever.}\\
\vspace{0.25cm}\\
\parbox{1\textwidth}{\hspace{0.75cm}I stepped into this heady stew of value when I joined Etsy in 2014. It was my second job, after a two year stint as a government contractor working in Texas on the IRS website, but it was the first where I felt like I was “making peoples’ lives better”. To a certain extent I was. Etsy had a couple hundred smart workers trying to build a place for people to make money doing what they loved. Our engineering team was smart, creative, and focused on the needs of Etsy’s sellers.}\\
\vspace{0.25cm}\\
\parbox{1\textwidth}{\hspace{0.75cm}In a perfect world, or maybe a nordic country, that might have been enough. We could have worked unburdened by the knowledge that every experiment we ran might have caused a seller to lose a sale that meant the difference between making rent or not. We could have ignored that many of our sellers worked day jobs in addition to selling on Etsy, and had limited access to healthcare. The interests of Etsy the company would be closely aligned with the interests of the sellers whose labor produced the goods that were sold on our marketplace, and not the investors sitting on the board of directors.}\\
\vspace{0.25cm}\\
\parbox{1\textwidth}{\hspace{0.75cm}The promise of a better future built with the power of tech isn’t a lie, exactly. The tools and technologies that we’ve developed to enrich ourselves are enormously powerful. Turned to a different end, perhaps they really could be used to make the world a better place. That’s what appeals to me so strongly about what Politics Rewired does. Organizing political power is a crucial first step in alleviating the systemic issues that’ve given rise to the gig economy. It shouldn’t be the exclusive purview of political elites. I’d like the dedicate my working days to making that a reality.}\\
\vspace{0.5cm}\\

\parbox{1\textwidth}{Sincerely,\\Kiron Roy}

\end{document}